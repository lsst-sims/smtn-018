\section{Introduction}



Throughout this paper we consider a Gen 2 Starlink satellite constellation as in \citet{Hu2022}. This constellation has 30,000 satellites in altitudes ranging from 340 to 614 km. 

This is the LSST overview paper: \cite{2008arXiv0805.2366I}.



\section{Pixel Impact}

After year 1, 8\% of all images would have at least one streak from the constellation. Per image, the mean streak length comes to 15.3 arcmin, so a 1 arcmin streak mask would results in 0.04\% of pixels being lost (assuming 0.2 arcsec per pixel and 3.2 Gpixels per visit).





\subsection{Toss a Snap}

We can consider a ``worst-case" scenario that if a visit contains a satellite streak, then that snap with the streak is discarded and a visit is reduced to a single 15s exposure rather than two 15s exposures. This would result in the 5-$\sigma$\ limiting depth being 0.37 mags shallower than if there was no streak.

In Figure~\ref{fig:depth_change}, we show the result for rejecting snaps with streaks for the first year of the baseline v3.0 survey in $r$. 

For this subset of exposures ($r$ band in the first year, only 30s visits so no twilight NEO observations), we find that of the 44,000 visits, 5\% would be streaked by an illuminated satellite. A large portion of the sky would be unaffected, with the mean change in coadded depth being 0.011 mags. 

\begin{figure}
\plottwo{plots/no_streak_map.pdf}{plots/streaked_map.pdf}
\plottwo{plots/mag_diff_map.pdf}{plots/mag_diff_hist.pdf}
\caption{ \label{fig:depth_change}}
\end{figure}



\section{Solar System Impact}


We can check how satellites could impact discovery of solar system objects. 