\section{Introduction}

Throughout this paper we consider a Gen 2 Starlink satellite constellation as in \citet{Hu2022}. This constellation has 30,000 satellites in altitudes ranging from 340 to 614 km. We pair that with the pointing history generated in the baseline 3.0 survey cadence simulation. This 10 year survey simulation has 2.08 million visits and includes high airmass visits in twilight time to search for inner solar system objects.

% This is the LSST overview paper: \cite{2008arXiv0805.2366I}.

\begin{figure}
\plotone{plots/n_streak_hist.pdf}
\caption{Number of images with varying number of streaks. \label{fig:streak_hist_1}}
\end{figure}



\begin{figure}
\plotone{plots/mean_streak_hist.pdf}
\caption{Mean number of streaks as a function of observing night and sun altitude. White vertical streaks are caused by weather and downtime. \label{fig:streak_hist_2}}
\end{figure}

\section{Pixel Impact}

After year 1, 8\% of all visits would have at least one streak from the constellation. Per image, the mean streak length comes to 15.3 arcmin, so a 1 arcmin streak mask would results in 0.04\% of all pixels being lost (assuming 0.2 arcsec per pixel and 3.2 Gpixels per visit).

Some of the most impacted observations are the high airmass observations taken in twilight to detect inner solar system objects. These visits consist of single 15s snaps. Table~\ref{table:pixel} breaks down the impact of streaks for twilight and non-twilight NEO visits. We expect the total pixels lost to scale with the size of the satellite constellation as well as with the streak mask width, with brighter satellites potentially needing larger masks. Note the number of visits with at least one streak does not scale with constellation size, as the Starlink constellation is in a relatively low orbit so the satellites are in earth's shadow for most of the night. Figure~\ref{fig:streak_hist_2} shows that images are only streaked when the sun is higher than $-30^\circ$\ altitude.

\begin{table}
\begin{tabular}{ l  c  c c}
  visits & N visits & at least  & total pixels lost \\
  & & one streak & with 1 arcmin mask \\
  \hline			
  all & 216120 & 8.4\%  & 0.043\% \\
  2x15s & 210679 & 7.4\% &  0.039\% \\
  1x15s (twilight NEO)  & 5441 & 45.7\% & 0.19\% \\
  \hline  
  \label{table:pixel}
 % 
\end{tabular}
\end{table}


\subsection{Toss a Snap}

We can consider a ``worst-case" scenario that if a visit contains a satellite streak, then the snap with the streak is completely discarded and the visit is reduced to a single 15s exposure rather than two 15s exposures. This would result in the 5-$\sigma$\ limiting depth being 0.37 mags shallower than if there was no streak.

In Figure~\ref{fig:depth_change}, we show the result for rejecting snaps with streaks for the first year of the baseline v3.0 survey in $r$. 

For this subset of exposures ($r$ band in the first year, only 30s visits so no twilight NEO observations), we find that of the 44,000 visits, 5\% would be streaked by an illuminated satellite. The majority of the sky would be unaffected, with the mean change in coadded depth being 0.011 mags. 

\begin{figure}
\plottwo{plots/no_streak_map.pdf}{plots/streaked_map.pdf}
\plottwo{plots/mag_diff_map.pdf}{plots/mag_diff_hist.pdf}
\caption{ Top Left: Coadded depth in $r$ after 1 year with no streak losses. Top Right: Coadded depth in $r$\ where snaps with streaks have been excluded. Bottom Left: Difference between the top images. Bottom Right: Histogram of the depth differences. \label{fig:depth_change}}
\end{figure}



\section{Solar System Impact}
% origianlly from repo at https://github.com/yoachim/22_Scratch/tree/main/neo_check

We can check how satellites could impact discovery of solar system objects. 

We use a sample of 10k objects in Vatira-like orbits (objects interior to the orbit of Venus) and find which objects could potentially be observed in the 10 year LSST survey. We then check which of those Vatira detections would intersect with Starlink streaks. For an extremely large 10 arcmin wide streak mask we find 5.7\% of the Vatira objects would hit a streak, and for a (still generous) 1 arcmin wide streak mask 0.53\% of Vatira observations would hit a streak (6,117 Vatira potential observations drops to 6,068). 

The Vatiras are the most sensitive to satellite streaks. When we repeat the experiment for PHA NEOs, only 0.07\% of potential detections are lost to a 1 arcmin streak. 

Note that these losses do not directly translate to losses in object identification, e.g., if there are 10 observations of an object, and one is lost to a satellite streak, we will probably still be able to fit it's orbit. 



