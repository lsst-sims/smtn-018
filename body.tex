\section{Introduction}

We construct a Gen 2 Starlink satellite constellation as in \citet{Hu2022}. xxx-basic stats of the constellation.



\section{Pixel Impact}

After year 1, 8\% of images would be streaked. Per image, the mean streak length comes to 15.3 arcmin, so a 60 arcsecond streak mask would result in 0.04\% of pixels peing lost.





\subsection{Toss a Snap}

We can consider a ``worst-case" scenario that if a visit contains a satellite streak, then that snap with the streak is discarded and a visit is reduced to a single 15s exposure rather than two 15s exposures. This would result in the 5-$\sigma$\ limiting depth being 0.37 mags shallower than if there was no streak.

In Figure~\ref{fig:depth_change}, we show the result for rejecting snaps with streaks for the first year of the baseline v3.0 survey in $r$. 

For this subset of exposures ($r$ band in the first year, only 30s visits so no twilight NEO observations), we find that of the 44,000 visits, 5\% would be streaked by an illuminated satellite. A large portion of the sky would be unaffected, with the mean change in coadded depth being 0.011 mags. 

\begin{figure}
\plottwo{plots/no_streak_map.pdf}{plots/streaked_map.pdf}
\plottwo{plots/mag_diff_map.pdf}{plots/mag_diff_hist.pdf}
\caption{ \label{fig:depth_change}}
\end{figure}



\section{Solar System Impact}


This is the LSST overview paper: \cite{2008arXiv0805.2366I}.

First paper on satellites \citet{Hu2022}
